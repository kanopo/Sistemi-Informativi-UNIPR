\section{Introduzione ai modelli}
Considerando un sistema reale(esistente o in fase di progetto),
non è possibili agire direttamente su di esso perchè troppo costoso(tempo e danaro).

Di solito si sfruttano rappresentazioni e modelli del sistema.

\subsection{Componenti dei modelli}
\begin{itemize}
    \item dati
    \item variabili decisionali
    \item vincoli
    \item obiettivo
\end{itemize}

\subsubsection{Dati}
Tutte le grandezze di un sistema reale \textbf{che non sono in nostro diretto controllo}.

\textbf{Esempio della casa}:
Il terrono sul quale la casa deve essere edificata

\subsubsection{Variabili decisionali}
Tutte le granddezze del sistema reale su cui abbiamo un controllo diretto.

A diverse configurazioni di variabili decisionali corrispondono diversi output.

\textbf{Esempio della casa}:
Grandezza delle stanze o colore pareti.

\subsubsection{Vincoli}
Solitamente bisogna rispettare alcuni vincolo, 
che inpongono limiti sui valori che possono assumere 
le variabili.

\textbf{Esempio della casa}:
Metratura minima della cucina, norme che inpongo certi criteri sulla sicurezza.

\subsubsection{Obiettivo}
Nel assegnare valori rispettando i vincoli, siamo guidati dall'\textbf{obiettivo}.

Vogliamo i scegliere i valori per arrivare all'obiettivo nella maniera più ottimale possibile.

\textbf{Esempio della casa}:
Spendere il meno possibilie.

\subsection{Procedura di risoluzione}
Dopo aver definito il modello, lo si deve risolvere.

La procedura di risoluzione sceglie i valori da assegnare alel variabili rispettando i vincoli e in modo da 
ottimizare il modello rispetto all'obiettivo.

\subsection{Validazione del modello}
Una volta ottenuta una configurazione ottimale, si esegue la validazione del modello,
il modello è una rappresentazione del sistema reale quindi bisogna chiedersi se la rappresentazione è fedele
oppure se si sono tralasciati alcuni aspetti.

\textbf{Esempio della casa}:
Ci siamo dimenticati del vincolo che serve almeno una finestra per ogni stanza, ecc.

La valutazione della risoluzione permette di trovare errori e di correggere il tiro.

\subsection{Tipi di modelli}
Tre macro gruppi:
\begin{itemize}
    \item modelli a scala: rappresentazione su scala ridotta
    \item modelli matematici: sistema tradotto in matematichese
    \item modelli di simulazione: sistema tradotto in informatichese
\end{itemize}

\subsubsection{Modelli matematici}
Vantaggio di poter usare tutti gli strumenti matematici nella risoluzione.

Attraverso questi è possibile algoritmi per la risoluzione.

Lo svantaggio dei modelli matematici è la limitata capacità espressiva.

\subsubsection{Modelli di simulazione}
Come vantaggi e svantaggi sono complementari ai modelli matematici, 
le procedure di risoluzione sono meno efficaci nel trovare la soluzione migliore 
ma hanno una capacità di espressione molto maggiore.

\subsection{Tipi di algoritmi}
Gli algoritmi trattati in questo corso sono:
\begin{itemize}
    \item \textbf{Algoritmi costruttivi}: si costruisce la soluzione ottimale partendo da soluzioni sub incomplete.
    \item \textbf{Algoritmi di raffinamento locale}: si raffinano algoritmi che possono già essere accettati ma sono sub ottimali.
    \item \textbf{Algoritmi di enumerazione}: si enumerano le possibili soluzioni per trovare quella migliore.
\end{itemize}

\subsubsection{Algoritmi costruttivi}
Per gli algoritmi di costruione c'è una sub classificazione:
\begin{itemize}
    \item \textbf{Algoritmi costruttivi senza revisione di decisioni passate}: i pezzi aggiunti non vengono più tolti
    \item \textbf{Algoritmi costruttivi con revisione di decisioni passate}: i pezzi aggiunti possono essere tolti
\end{itemize}

\subsubsection{Algoritmi di raffinamento locale}

Qui introcuciamo il concetto di "vicinanza", in genere per soluzione vicina, si intende
una soluzione che differisce solo in parte da quella attuale.

Di solito si vuole che la soluzione vicina sia migliore della soluzione attuale rispetto all'obiettivo prefissato.

\subsubsection{Algoritmi enumerativi}
Si dividono in:
\begin{itemize}
    \item \textbf{Algoritmi di enumerazione completi}: dove si valuta ogni singola soluzione ammissibile.
    \item \textbf{Algoritmi di enumerazione implicita}: dove attraverso la risoluzione di sottoproblemi, si enumerano implicitamente interi sottoinsiemi della regione ammissibile.
\end{itemize}

