\section{I principi contabili di base}
\subsection{Omogeneità}
Le registrazioni contabili si riferiscono solo a eventi che producono qualcosa quantificabile in termini monetari.

I termini inseriti sono in termini di potere d'acquisto della moneta in quel momento storico
(c'è da fare un'interpretazioni con il passare degli anni).


\subsection{Entità}

La contabilità si riferisce ad un'entità e non alle persone ad essa collegate.


\subsection{La prospettiva di continuità di funzionamento}
Si deve assumere che un'azienda continui all'infinito.

In caso si voglia chiudere si fa un bilancio di liquidazione e non un bilancio di esercizio.s


\subsection{Costo}
Un'attività è rilevata in contabilità al suo prezzo d'acquisto cioè al \textbf{costo storico}.

Le attività poi si possono dividere in monetarie e non, le monetarie hanno un'informazione
oggettiva sul valore (\textbf{fair value}), invece, le attività che non hanno un valore oggettivo(terreni, fabbricati, macchinari) per queste si tiene in considerazione il \textbf{costo storico(prezzo di acquisto iniziale)}.

Ovviamente il costo storico nel corso degli anni viene modificato a scendere.



\subsection{Duplice aspetto}

L'attività è la somma di passività e capitale netto
\subsection{Periodo della misurazione}
Si cerca di avere una periodicità nelle misurazioni in modo da capire l'andamento dell'azienda
e aggiustare il tiro.

Il periodo amministrativo va dal 1 Gennaio al 31 Dicembre.


\subsection{Prudenza}
Si devono trattare i dati con ragionevole scetticismo di modo da aumentare la credibilità dei risultati.


\textbf{prudenza}: attitudine a sottostimare il reddito e le attività
qualora sussita incertezza.

Applicando la prudenza si ha che:
\begin{itemize}
    \item I ricavi(aumento utili) si riconoscono solo quando sono ragionevolmente certi
    \item riconoscere i costi(diminuzione di utili) non appena sono ragionevolmente possibili
\end{itemize}

I ricavi sono normalmente riconosciuti all consegna del prodotto al cliente.



\subsection{Realizzazione dei ricavi}

Quanto ricavo devo riconoscere??

Quello che il cliente con ragionevole certezza pagherà.


\subsection{Competenza}

Un costo di competenza di un certo periodo è un costo da associare a
quel periodo amministrativo, rappresenta risorse consumate nel
periodo per la produzione dei ricavi del periodo.

Sono necessarie operazioni di rettifica.


\subsection{Continuità dei criteri di valutazione}

Una volta adottato un metodo di valutazione devo rimanere con quello, cosi da evitare di fare conversioni tra metdoti e introdurre errori.

In questo modo è anche possibile confrontare bilanci di periodi diversi con facilità.


\subsection{Significità e rilevanza}

\begin{itemize}
    \item Trascurare le transazioni irrilevanti
    \item individuare l transazioni rilevanti
\end{itemize}

Sono rilevanti le transazioni che, se fossero contabilizzate, indurrebbero a valutare diversamente il bilancio.

