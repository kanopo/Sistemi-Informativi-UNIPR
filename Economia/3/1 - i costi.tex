\section{I costi}

\subsection{L'attività d direzione nelle imprese}

Dirigere significa prednere dicisioni per garantire l'efficacia e l'effcienza ai processi che formano la combinazione produttiva.

\subsection{Contabilità generale VS analitica}

\begin{figure}[H]
    \centering
    \includegraphics[width=0.7\linewidth]{3/img/contabilità generale vs analitica.png}
\end{figure}

\subsection{Il concetto di costo}

Valore monetario necessario epr lo svolgimento di un'operazione.


Si possono fare calcoli di costo finali che coprono un'intero processo produttivo oppure 
si calcola l'oogetto intermendio cioè ogni singola fase.

\subsection{Diverse connfigurazioni di costo}
\begin{itemize}
    \item costo primo
    \item costo di trasformzaione
    \item costo perno di produzione
    \item costo pieno aziendale
\end{itemize}

\subsection{Diverse classi di costo}
\begin{itemize}
    \item costi diretti e costi indiretti
    \item costi variabili e costi fissi
    \item costi comuni e costi congiunti
    \item costi di prodotto e costi di periodo
\end{itemize}

\subsection{Costi diretti e indiretti}
\subsubsection{Costi diretti}
Costi dei fattori produttivi utilizzati in via esclusiva per l'ottenimento di un prodotto.


\subsubsection{Costi indiretti}
Costi di fattori produttivi (generalemente strutturali) usati alternativamente o contemporaneamente
per la produzione di più prodotti.

\subsection{Costi variabilie e costi fissi}
\subsubsection{Costi variabili}
Costi che variano al variare dei volumi di produzione.

Se questa variazione non cresce in modo lineare ma ha una variabilità crescente o decrescente, 
prende il nnome di cvu(costo variabile unitario) crescente o decrescente.

\subsubsection{Costi fissi}
Costi che non cambinao in base al volume produttivo come l'affitto del capannone o robe di questo tipo.

\subsection{Intervallo di rilevanza}
È l'intervallo di attività o di volume all'interno del quale si suppone
valida una specifica relazione fra il livello di attività/volume e il costo.


\subsection{Periodo temporale di rilevanza}
I costi sotenibili dipendono dalla finestra temporale:
\begin{itemize}
    \item brevissimo periodo: quasi tutti i costi non sono modificabili
    \item breve/medio periodo: alcuni costi modificabili
    \item lungo periodo: quasi tutti i costi modificabili
\end{itemize}

\subsection{Stima della relazione costo-volume}
CT(costo totale) = CFT(Costo fisso totale) + cvu(costo variabile unitario) * X(numero di prodotti)

\subsection{Costi eliminabili e costi ineliminabili}
I costi eliminabili sono quesi costi che se un prodotto dovesse essere eliminato dalla produzione verrebbero meno.

I costi non eliminabili sono uesi costi che anche se un prodotto venisse eliminato non scomparirebbero.

\subsection{Le configurazioni di costo}
\subsubsection{Direct cost}
Il costo di prodotto dipende solo dai costi diretti.

\subsubsection{Full cost(o costo pieno)}
Il costo di prodotto è composto da costi diretti e da quote di costi indiretti attribuiti utilizzando 
delle basi di ripartizione.

\subsubsection{Calcolo costo unitario di prodotto: costi diretti}
Prezzo di aqusto del fattore produttivo * quantità del fattore produttivo consumata per unità di prodotto = costo diretto dell'unità di prodotto.

\subsubsection{Calcolo costo unitario di prodotto: costi indiretti}

Costo indiretto / base di ripartizione = coefficiente di attribuzione.


Coefficiente di attribuzione * quota della base di ripartizione consumata dall'unità di prodotto = quota di costo indiretto per unità di prodotto.

\subsection{Orientamento ai fattori produttivi}
Le singole voci di costo sono raggruppate per categorie di fattori produttivi come:
\begin{itemize}
    \item costo alvoro indiretto
    \item ammortamenti industriali
    \item ammortamenti non industriali
    \item costo lavoro non industriale
    \item affitto e costi di struttura
    \item ecc
\end{itemize}

\subsection{Orientamento funzionale}
stessa cosa d quella sopra ma si aggregano i costi in base a:
\begin{itemize}
    \item costi indiretti di produzione
    \item costi commerciali
    \item costi amministrativi
    \item costi generali
    \item ecc
\end{itemize}

\subsection{Gerarchia centri di costo}
\begin{itemize}
    \item centri di produzione: processi di trasformazione
    \item centri ausiliari: fornisce robe agli altri centri di costo
    \item centri di servizi: esternia all'area produttiva(area comemrciale, amministrativa, ecc)
    \item centri virtuali: costi residuali per far quadrare i conti
\end{itemize}

\subsubsection{Medoto diretto}
si ipotizza una relazione diretta di ciascun centro
di servizi con i centri di costo di produzione.

Il pregio di questo metodo è rappresentato dalla semplicità, ma presenta
il grande limite di non considerare le relazioni tra centri di servizi.


\subsubsection{Metodo dei passaggi}
Esprime il legame tra alcuni centri di servizi.

Questo metodo permette di trattare in cascata i vari costi.

\subsubsection{Metodo reciproco}
Crea allocazioni reciproche e permette di trovare le interazioni tra i vari centri di costo.

Si fa un sistema di equazioni.

\section{Da finire esercizi da pag 105 metalmec}