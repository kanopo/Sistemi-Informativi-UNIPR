\section{Costi}
\subsection{Determinazione dei costi di prodotto nelle aziende su commessa}
cda(coefficiente di allocazione) = costi indiretti da allocare / bdr(base di ripartizione)

Per sapere la quota di costi indiretti da allocare:
cda * valore della bdr per la commessa


\subsection{Costi di prodotto nelle aziende che operano per processo}
costo medio unitario di prodotto = CT(costo totale) di reparto / unità prodotte


UEP (unita equivalente di produzione) = n unità completate + n unità non completate * percentuale di completamento


\subsection{Costi di prodotto nelle aziende che operano per lotti}


si calcola un coefficiente di allocazione per ogni stazione di lavorazione che attraversa.

\subsection{Direct costing e margine di contribuzione}

\begin{itemize}
    \item margine di contribuzione unitario: ricavi unitari - costo di prodotto
    \item margini di contribuzione complessivo: margine di contribuzione unitario * prodotti venduti
    \item margine di contribuzione aziendale: sommatoria dei margini di contribuzione 
\end{itemize}



Il profitto è: ricavi - costi variabili - costi fissi.


\subsection{Variable costing}
costo unitario variabile industriale = costo primo + costi diretti di trasformazione + costi indiretti di produzione


costo uintario variabile commerciale = costi correlati ai volumi di vendita(provvigioni)


costo unitario variabile aziendale = costo unitario variabile industriale + costo unitario variabile commerciale + eventuali costi aziendali



