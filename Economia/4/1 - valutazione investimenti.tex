\section{Investimenti}
\subsection{che cosa è un investimento}



Gli investimenti sono uscite di cassa, in una o più soluzioni
(fase di impianto), che genereranno nuovi flussi di cassa
positivi (fase di esercizio);

\subsection{Principali tipologie di investimenti
industriali}

\begin{itemize}
    \item espansione(acquisto nuovo stabilimento)
    \item ampliamento dell'offerta(nuovi prodotti)
    \item sostituzione(rimpiazzo impianto)
    \item automatizzazione/digitalizzazzione per ridurre i costi e migliorare la qualità
    \item scelta tecnologica
\end{itemize}


\subsection{Considerando il tempo}
valore attuale = valore futuro * coefficiente di attualizzazione($\frac{1}{1 + r}$) dove r è il coefficente che si considera


Valore futuro = valore attuale * coefficiente di capitalizzazione($1 + r$)


\subsection{Il montante di un singolo flusso dopo n
periodi}

Montare = valore attuale * $(1 + r)^n$ dove n sono i perdiodi


\subsection{Le principali tecniche di valutazione
economico-finanziaria degli investimenti}

\begin{itemize}
    \item PBP (payback period, tempo di recupero)
    \item dPBP (discounted pay back period, tempo di recupero
    attualizzato)
    \item VAN (valore attuale netto)
    \item TIR (tasso interno di rendimento)
    \item IRA (indice di rendimento attualizzato)
\end{itemize}

\subsubsection{Il tempo di recupero attualizzato (dPBP)}

tiene conto del valore finanziario dell'investimento nel tempo

\begin{equation*}
    \sum_{t = 1}^{PBP} \frac{F_{(t)}}{(1 + r)^t} - f_0 = 0
\end{equation*}

\subsubsection{Il valore attuale netto (VAN)}
il VAN è la somma algebrica di tutti i flussi di cassa
attualizzati.

se il VAN è positivo c'è creazione di valore !


\begin{equation*}
    VAN = \sum_{t = 1}^{n} \frac{F_t}{(1+r)^t} - F_0
\end{equation*}

\subsubsection{Il tasso interno di rendimento (TIR)}

tasso che rende il VAN = 0.

Quindi faccio formula del VAN = 0

Il TIR è il costo massimo che un progetto può sopportare affinche sia economicamente conveniente.


Se TIR maggiroe di 0 allora da ufo, altrienti merdata.

\subsubsection{L’indice di rendimento attualizzato (IRA)}
\begin{equation*}
    IRA = \frac{
        \sum_{t = 1}^{n} F_t \cdot \frac{1}{(1 + r)^t}
    }{
        F_0
    }
\end{equation*}

Praticamente la somma di tutti i flussi attualizzati ritornati dall'investimento divisi per l'investimento iniziale.

\subsection{Procedimento per la valutazione degli
investimenti}

\begin{enumerate}
    \item calcola i flussi di cassa operativi per gli anni dell'investimento(si parte da 0 e si arriva a n)
    \item determinare il tasso di attualizzazione r
    \item calcola il coeffieciente di attualizzazione per gli n anni di vita dell'investimento
    \item attualizzare i flussi di cacssa operativi per ogni anno
    \item calcolare il VAN per decidere se conviene
    \item calcolare TIR e dPBP
\end{enumerate}