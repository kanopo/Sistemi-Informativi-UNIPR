\section{Introduzione all'elettronica}
\subsection{Trasduttori}
I trasduttori sono dispositivi che mettono in contatto la realtà e l'elettonica.

ne esistono di due famiglie:
\begin{itemize}
    \item sensori
    \item trasduttori
\end{itemize}

I sensori trasformano grandezze fisiche in elettriche, mentre i trasduttori utilizzano le grandezze elettriche per trasformarle in grandezzze fisiche.

\subsection{Digitale vs analogico}
\begin{itemize}
    \item grande potenza di calcolo ed eleaborazione del segnale
    \item maggior robustezza ai disturbi
    \item minor sensibilità alla temperatura
\end{itemize}

\subsection{Analogico vs digitale}
\begin{itemize}
    \item in natura le grandezze fisiche sono descrivibili come segnali analogici
    \item sensori e attuatori
    \item per la conversione da analogico a digitale e viceversa, si usano circuiti DAC e ADC
\end{itemize}

\subsection{ADC(Convertitore analogico digitale)}
\begin{itemize}
    \item viene fissata la tensione di fondo scala ($V_{fs}$)
    \item la tensione d'ingresso analogica viene convertita nel valore più vicino numero a n-bit
    \item maggiore è il numero di bit usati per la conversione e maggiore è la precisione del ADC(si perdono meno informazioni nella conversione)(minor errore di quantizzazione).
\end{itemize}


\subsection{DAC(convertitore digitale analogico)}
la tensione in uscita è:
\begin{equation*}
    V_O = (\sum_n^{+\infty} b_n 2^{-n}) V_{fs}
\end{equation*}
\begin{equation*}
    V_O = (b_12^{-1} + b_22^{-2} + \dots + b_n2^{-n}) V_{fs}
\end{equation*}

Scritto in due maniere(sero uguali)

