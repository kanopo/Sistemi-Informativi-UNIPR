\section{Semiconduttori}

\subsection{Caratteristiche}
\begin{itemize}
    \item Resistività ($\rho$) intermedia tra isolanti e conduttori
    \item possibilità di variare $\rho$ mediante il drogaggio
    \item due portatori di carica(elettroni e lacune)
\end{itemize}

\subsection{Drogaggio di un semiconduttore}
Sostanzialmente si mettono atomi di diverso tipo nel composto che va a formare il semiconduttore finale.

Quando parliamo di silicio, distinguiamo silicio-p e silicio-n.
\begin{itemize}
    \item droganti di tipo n: elementi del 5 gruppo(5 elettroni esterni o di valenza)
    \item droganti di tipo p: 3 elettroni di valenza 
\end{itemize}


Nei composti drogati di tipo n, si forma un atomo libero di muoversi e nei composti di tipo p si ha una mancanza di un atomo(quidni una lacuna).


\subsection{Correnti di "drift"}

