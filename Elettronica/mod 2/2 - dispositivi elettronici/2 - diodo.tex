\section{Diodo}
\subsection{Diodo a giunzione PN}

Il diodo è formato da due parti, una parte drogata di tipo p e una drogata di tipo n.

Questa costruzione forma ua forza che impedisce il passaggio di cariche nei diodi perchè si oppone 
a quello che dovrebbe essere il normale fluire delle  cariche avendo una parte positiva e una negativa.

Fra la parte P e quella N si trova la "regione svuotata".

\subsection{Diodo polarizzato in diretta}

la tensione applicata dall'esterno si localiza tutta ai capi della regione svuotata, la corrente di diffusione prevade su quella di drift e c'è il passaggio delle cariche.


Tendenzialmente la tensione da vincere per permettere il fluire della corrente è di $V_T = 0.7V$ ma varia in base ai materiali impiegati e alle temperature.

\subsection{Diodo polaizzato in inversa}


La barrriera di potenziale si alza un \textbf{botto}.

Per correnti molto la giunzione funziona come interruttore aperto, però se si continua ad
aumentare la corrente si incombe nella corrente di breakdown.

\subsubsection{Breakdown a valanga}
\begin{itemize}
    \item vengono iniettati elettroni nella regione svuotata
    \item forte campo elettrico nella regione svuotata
    \item l'elettrone acquista elevata forza cinetica
    \item collisione con atomo nel reticolo
    \item un'elettrone dell'atomo si libera e applica energia ad altri legami
    \item si crea una valanga dove esplode tutto
    \item il diodo è cafuddato
\end{itemize}

