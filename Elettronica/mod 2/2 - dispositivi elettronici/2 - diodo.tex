\section{Diodo}
\subsection{Diodo a giunzione PN}

Il diodo è formato da due parti, una parte drogata di tipo p e una drogata di tipo n.

Questa costruzione forma ua forza che impedisce il passaggio di cariche nei diodi perchè si oppone 
a quello che dovrebbe essere il normale fluire delle  cariche avendo una parte positiva e una negativa.

Fra la parte P e quella N si trova la "regione svuotata".

\subsection{Diodo polarizzato in diretta}

la tensione applicata dall'esterno si localiza tutta ai capi della regione svuotata, la corrente di diffusione prevade su quella di drift e c'è il passaggio delle cariche.


Tendenzialmente la tensione da vincere per permettere il fluire della corrente è di $V_T = 0.7V$ ma varia in base ai materiali impiegati e alle temperature.

\subsection{Diodo polaizzato in inversa}


