\section{Metodo dei nodi}

\subsection{Leggi di Kirchhoff}
\subsubsection{Prima L.D.K delle correnti(KCL)}
\begin{quote}
    La somma algebrica delle correnti dei rami convergenti in un nodo è sempre nulla.
\end{quote}


\subsubsection{Seconda L.D.K delle tensioni(KVL)}
\begin{quote}
    La somma algebrica delle tensioni lungo una maglia è sempre nulla.
\end{quote}

\subsection{Metodo dei nodi}

\begin{enumerate}
    \item Identificare i nodi e fra questi decidere quale è il nodo di riferimento
    \item Identificare i versi (in modo arbitrario) delle correnti di ciascun ramo
    \item Scrivere le equazioni costitutive dei modelli dei componenti per esprimere le correnti di cui al punto 2 in funzione dei soli potenziali fra i nodi
        \begin{enumerate}
            \item se ancora vi sono correnti non in funzione dei potenziali, applicare KVL
        \end{enumerate}

    \item Risolvere il sistema che ha come incognite i soli potenziali di nodo
    \item Se necessario, determinare le altre tensioni con la KVL e le correnti usando le eqauzioni dei componenti
\end{enumerate}


