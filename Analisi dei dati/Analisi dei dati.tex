\documentclass{article}
\usepackage{graphicx}
\usepackage{float}
\usepackage{amsmath}
\usepackage{amsfonts}
\usepackage{amssymb}
\usepackage{hyperref}
\usepackage{esint}
\usepackage[utf8]{inputenc}
\usepackage[a4paper, portrait, margin=0.75in]{geometry}
\setlength\parindent{0pt}
\usepackage[italian]{babel}





\hypersetup{
    colorlinks=true,
    linkcolor=black,
    filecolor=magenta,
    urlcolor=blue,
    pdftitle={Tecnologie internet},
    pdfpagemode=FullScreen,
}


\begin{document}
    \author{kanopo}
    \title{Analisi dei dati}
    \date{2022}

    \maketitle
    \tableofcontents

    \listoffigures
    \listoftables

    \section{Statistica descrittiva}

    \subsection{Le grandezze che sintetizzano i dati}

    \subsubsection{Media}

    Dato un inseme $x_1, x_2, \dots, x_n$ di dati, la media è la media aritmetica:
    \begin{equation*}
        \bar{x} = \frac{1}{n} \sum_{i=1}^{n} x_i
    \end{equation*}

    \subsubsection{Mediana}
    Dato un'insieme di dati di ampiezza n, lo si ordina dal minore al maggiore. La mediana è il valore che
    occupa la posizioe $\frac{n+1}{2}$ in caso di un'insieme dispari, oppure
    la media fra $\frac{n}{2}$ e $\frac{n+1}{2}$ se pari.

    \subsubsection{Moda}
    La moda campionaria di un'insieme di dati, se esiste, è l'unico valore che ha la frequenza massima.


    \subsubsection{Varianza e deviazione standard}

    Dato un'insieme di dati $x_1, x_2, \dots, x_n$, si dice varianza campionaria ($s^2$), la quantità 

    \begin{equation*}
        s^2 = \frac{1}{n-1} \sum_{i=1}^{n} (x_i - \bar{x})^2
    \end{equation*}

    Una comodità per il calcolo è
    \begin{equation*}
        \sum_{i=1}^{n} (x_i - n\bar{x})^2 = \sum_{i=1}^{n} x_i^2 - n\bar{x}^2
    \end{equation*}

    Si dice \textbf{deviazione standard campionaria} e si denota  con $s$, la quantità
    \begin{equation*}
        s = \sqrt{\frac{1}{n-1} \sum_{i=1}^{n} (x_i - \bar{x})^2}
    \end{equation*}



    \subsubsection{Percentili campionari e box plot}


    Sia $k$ un numero intero $0 \leq k \leq 100$.

    Dato un campione di dati, esiste sempre un dato che è contemporaneamente maggiore del $k$ percento dei dati e minore del $100-k$ percento.




















































































\end{document}