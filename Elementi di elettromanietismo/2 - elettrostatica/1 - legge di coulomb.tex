\section{Legge di Coulomb}
\subsection{Struttura atomica}
La carica dell'elettrone è $e=1.60218 \times 10^{-19}C$(Coulomb)

\subsection{Quantizzazione della carica}
La carica del protone è $+e$, la carica dell'elettrone è $-e$ ed il neutrone ha carica 0.
La carica elettrica di un qualunque oggetto è:
\begin{equation}
    q = (N_p-N_e)\times e
\end{equation}

\subsection{Legge di Coulomb}
Interzione elettriva tra due particelle.
\begin{itemize}
    \item $a$ possiede carica $q_a$ e si trova nell'origine del sistema
    \item $b$ possiede carica $q_b$ e si trova a una distanza $r$ da $a$
\end{itemize}
La forza esercitata da $a$ su $b$ è:
\begin{equation}
    \vec{F_{ab}} = \frac{1}{4\pi \epsilon_0}\frac{q_aq_b}{r^2}\hat{r}
\end{equation}

La forza di Coulomb è la forza che agisce tra particelle cariche.


Ricorati che essedo forze hanno un verso.

\begin{itemize}
    \item $\vec{F_{ab}}$ è la forza esercitata da $a$ su $b$
    \item $\vec{F_{ba}}$ è la forza esercitata da $b$ su $a$
\end{itemize}

Questa forza è inversamente proporzionale al quadrato della distanza(se distanza x2, intensità /4).
Le particelle con la stessa carica si respingono e quelle di carica opposta si attraggono.

La forza di Coulomb è vettoriale e obbedisce alla terza legge di Newton($F_{ab} = -F_{ba}$).

La costante di proporzionalità $\frac{1}{4\pi \epsilon_0}$ vale:
\begin{itemize}
    \item $\epsilon_0 = 8.854\times 10^{-12} [\frac{C^2}{N m^2}]$
    \item $\frac{1}{4\pi \epsilon_0} = 9\times 10^{9} [\frac{Nm^2}{C^2}]$
\end{itemize}

\subsection{Principio disovrapposizione}
Se sono presenti più cariche, gli effetti delle singole forze si sommano fra di  loro essendo vettoriali.

\begin{equation}
    \vec{F} = \vec{F_1} + \vec{F_2} + \dots + \vec{F_n}
\end{equation}

\begin{equation}
    \vec{F} = \frac{1}{4\pi \epsilon_0}\Big(\frac{qq_1}{r^2_1}\hat{r_1} + \frac{qq_2}{r^2_2}\hat{r_2} + \dots + \frac{qq_n}{r^2_n}\hat{r_n}\Big)
\end{equation}