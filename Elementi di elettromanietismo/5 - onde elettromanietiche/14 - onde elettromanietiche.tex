\section{Onde elettromanietiche}
\begin{equation}
    c = \frac{1}{\sqrt{\epsilon_0\mu_0}}
\end{equation}
Questa è una costante i cui valori sono:
\begin{itemize}
    \item $c = 2.99792458\times 10^8 [\frac{m}{s}]$
    \item $\mu_0 = 4\pi \time 10^{-7} [\frac{H}{m}]$
    \item $\epsilon_0 = 8.85 \times 10^{-12} [\frac{F}{m}]$
\end{itemize}

Con costante dielettrica relativa $\epsilon_r = \epsilon/\epsilon_0$
e permeabilità magnetica $\mu_r = \mu/\mu_0$ si ottiene:
\begin{equation}
    v = \frac{1}{\sqrt{\epsilon_0\epsilon_r\mu_0\mu_r}} = 
    \frac{1}{\sqrt{\epsilon_0\mu_0}}\frac{1}{\sqrt{\epsilon_r\mu_r}} = 
    \frac{c}{\sqrt{\epsilon_r\mu_r}}
\end{equation}

Le formule comode per la velocità sono queste:
\begin{equation}
    v = \frac{c}{\sqrt{\epsilon_r\mu_r}}
\end{equation}
Le formule comode per la rifrazione:
\begin{equation}
    n = \frac{c}{v}
\end{equation}

\subsection{Caratteristiche delle onde EM piane}

Se ho il campo magnetico o elettrico che polarizza l'onda:
\begin{equation}
    E_0 = cB_0
\end{equation}

E per le altre formule basarmi sulle formule delle onde.

\subsection{Intensità delle onde elettromanietiche}

Densità di energia associata al campo elettrico:
\begin{equation}
    u_E = \frac{1}{2}\epsilon_0E^2
\end{equation}
Densità di energia associata al campo magnetico:
\begin{equation}
    u_B = \frac{1}{2}\frac{B^2}{\mu_0}
\end{equation}

Le onde elettromanietiche piane hanno l'energia magnetica e elettrica uguale,
dato che per calcolare l'energia elettromaietica devi sommare le due,
puoi limitarti a calcolare solo una delle due:
\begin{equation}
    u_{e.m.} = u_E + u_B = 2u_B = 2u_E = \epsilon_0E^2 = 
    \frac{B^2}{\mu_0}
\end{equation}

L'intensità dell'onda è circa:
\begin{equation}
    I = A^2
\end{equation}
circa il quadrato dell'ampiezza.

Per un'onda elettromaietica:
\begin{equation}
    S = u_{e.m.}c = \epsilon_0E^2c = \frac{E^2}
    {\sqrt{\frac{\mu_0}{\epsilon_0}}}
\end{equation}
Che è l'intensità di un'onda elettromaietica.

